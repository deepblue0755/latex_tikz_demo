% A hexagon for memorizing trigonometric identities
% Author: Josef Nilsen
% \documentclass{article}
\documentclass[border=5pt]{standalone}
\usepackage{tikz}
%%%<
\usepackage{verbatim}
\usepackage[active,tightpage]{preview}
\PreviewEnvironment{tikzpicture}
\setlength\PreviewBorder{5pt}%
%%%>
\begin{comment}
:Title: A hexagon for memorizing trigonometric identities
:Tags: Foreach, Paths,Mathematics
:Author: Josef Nilsen
:Slug: trigonometric-hexagon
\end{comment}
\begin{document}
\begin{tikzpicture}[scale=4,cap=round,>=latex]
% Radius of regular polygons
  \newdimen\R
  \R=0.8cm
  \coordinate (center) at (0,0);
 \draw (0:\R)
     \foreach \x in {60,120,...,360} {  -- (\x:\R) }
              -- cycle (300:\R) node[below] {$\csc \theta$}
              -- cycle (240:\R) node[below] {$\sec \theta$}
              -- cycle (180:\R) node[left] {$\tan \theta$}
              -- cycle (120:\R) node[above] {$\sin \theta$}
              -- cycle (60:\R) node[above] {$\cos \theta$}
              -- cycle (0:\R) node[right] {$\cot \theta$};
  \draw { (60:\R) -- (120:\R) -- (center) -- (60:\R) } [fill=gray];
  \draw { (180:\R) -- (240:\R) -- (center) -- (180:\R) } [fill=gray];
  \draw { (0:\R) -- (300:\R) -- (center) -- (0:\R) }  [fill=gray];
   \R=0.1cm
  \draw (0:\R) \foreach \x in {60,120,...,360} { -- (\x:\R) }
    [fill=white] -- cycle (center) node {1};
\end{tikzpicture}
\end{document}
