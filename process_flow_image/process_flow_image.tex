\documentclass[10pt]{article}
\usepackage{geometry}

\usepackage[english]{babel}
\usepackage[utf8]{inputenc}
\usepackage{tikz}
\usetikzlibrary{arrows.meta,
                chains,
                fit,
                positioning,
                shapes.arrows}

\usepackage[floats,active,tightpage]{preview}
\setlength\PreviewBorder{1em}

\usepackage{showframe}

\begin{document}

\begin{figure}[h]
    \centering
\begin{tikzpicture}
[
node distance = 10mm and 6mm,
  start chain = A going right,
 force/.style = {rectangle, rounded corners, draw, fill=cyan!30,
                 inner sep=2mm, outer sep=0mm, minimum size=8mm,
                 font=\bfseries\sffamily, on chain},
    CA/.style = {% Connection Arrow
                 single arrow, draw,
                 single arrow head extend=1.5mm,
                 minimum height=6mm, minimum width=5mm, outer sep=0mm},
    LA/.style = {% Long Arrow
                 CA, draw=none, left color=cyan!20, right color=cyan,
                 inner xsep = 6mm, minimum width=9mm, label=center:#1,},
 ]
    \begin{scope}[every node/.style={force}]
\node   {Planejar};     % A-1
\node   {Adquirir};
\node   {Processar};
\node   {Analisar};
\node   {Preservar};
\node   {Publicar};     % A-6
    \end{scope}
% arrows between nodes
\foreach \i in {1,...,5}
    \node[CA, right=0mm of A-\i] {};
% long arrows with text
\coordinate[below=of A-1]   (a1);
\coordinate[below=of a1]    (a2);
\coordinate[below=of a2]    (a3);
    \node[LA=Metadados e Documentação,
          fit=(a1) (a1 -| A-6)] {};
    \node[LA=Gestão da Qualidade,
          fit=(a2) (a2 -| A-6)] {};
    \node[LA=Disponibilidade e Segurança dos Dados,
          fit=(a3) (a3 -| A-6)]    {};
\end{tikzpicture}
\caption{Modelo de ciclo de vida dos dados de pesquisa proposto pela USGS. Adaptado de}
\label{fig:modeloUSG}
\end{figure}

\end{document}
