\documentclass[varwidth=true, border=10pt]{standalone}
\usepackage{tkz-euclide}
\newcommand{\iu}{{i\mkern1mu}} % imaginary unit

\begin{document}
\usetkzobj{all}
\begin{tikzpicture}[scale=3]
    \tkzSetUpPoint[shape=circle,size=3,color=black,fill=black]
    \tkzSetUpLine[line width=1]
    \tkzInit[xmax=1.2,ymax=1,xmin=-1.2,ymin=0]
    \pgfmathsetmacro{\Radius}{1}
    \tkzDefPoints{2.0/1.5/Z, 0/0/O, 0/1/i}

    %% Konstruktion von 1/ \overline{z} und -1/ \overline{z}
    \tkzTangent[from with R = Z,/tikz/overlay](O,\Radius cm)  \tkzGetPoints{T1}{T2}
    \tkzInterLL(T1,T2)(O,Z) \tkzGetPoint{dZ}
    %%

    \tkzDrawArc[R,line width=1pt,color=orange](O,\Radius cm)(0,180)
    \tkzMarkAngle[size=1mm](Z,dZ,T1)
    \tkzLabelAngle[pos=0.06](Z,dZ,T1){$\cdot$}
    \tkzAxeXY

    \tkzDrawPoints(Z, dZ, T1)
    \tkzLabelPoint[above left](Z){$z = r \cdot e^{\iu \varphi}$}
    \tkzLabelPoint[below right](dZ){$\frac{1}{\overline{z}} = \frac{1}{r} \cdot e^{\iu \varphi}$}
    \tkzDrawSegments[dashed](O,Z)
    \tkzDrawLine[dashed, add=0 and 0.5](Z,T1)
    \tkzDrawSegments[dashed](T1,dZ)
\end{tikzpicture}
\end{document}
