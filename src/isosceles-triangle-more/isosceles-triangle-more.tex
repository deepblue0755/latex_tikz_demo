\documentclass{article}

\usepackage{xcolor}
\definecolor{cWinkelhalbierende}{HTML}{00940A}
\definecolor{cSeitenhalbierende}{HTML}{FF0000}
\definecolor{cAchse}{HTML}            {94008A}
\definecolor{cMittelsenkrechten}{HTML}{181FED}

\usepackage[pdftex,active,tightpage]{preview}
\setlength\PreviewBorder{2mm}
\usepackage{tikz}
\usetikzlibrary{shapes, calc}

\begin{document}
\begin{preview}
\begin{tikzpicture}
    \coordinate[label=left:$A$]  (A) at (0,0);
    \coordinate[label=right:$B$] (B) at (3.2,0);
    \coordinate[label=left:$C$] (C) at (1.6,4);

    % Draw the background of the triangle
    \node[isosceles triangle, isosceles triangle apex angle=44,draw,
        inner sep=0pt,anchor=lower side,rotate=90,draw=black,
        line width=1.5pt, minimum height=4cm, fill=gray!2]
        (triangle) at (1.6,-0.05) {};

    % Rechte Winkel
    \begin{scope}[shift={($(A)!0.5!(C)$)}, rotate=-22]
        \draw[fill=gray!30] (0,0) -- (-180:0.25cm) arc (180:90:0.25cm);
        \draw (140:0.15cm) node {$\cdot$};
    \end{scope}

    \begin{scope}[shift={($(B)!0.5!(C)$)}, rotate=-70]
        \draw[fill=gray!30] (0,0) -- (-180:0.25cm) arc (180:90:0.25cm);
        \draw (140:0.15cm) node {$\cdot$};
    \end{scope}

    \begin{scope}[shift={($(A)!0.5!(B)$)}]
        \draw[fill=gray!30,draw opacity=0] (0,0) -- (-180:0.25cm) arc (180:90:0.25cm);
        \draw (-180:0.25cm) arc (180:90:0.25cm);
        \draw (140:0.15cm) node {$\cdot$};
    \end{scope}

    % Mittelsenkrechten
    % y = m * x + t
    % Für die Mittelsenkrechte m_AB gilt ($(A)!0.5!(B)$) ist auf m_AB
    % m = - (delta x) / (delta y) = - (A.x - B.x) / (A.y - B.y)
    % t = y - m*x

    % AB
    \draw[color=cAchse, thick] (1.6cm,-0.3) -- node {} (1.6cm, 4.2cm);

    % AC
    % (0.8, 2) \in  Mittelsenkrechte auf b
    % m = - (-1.6)/(-4) = - 0.4
    % t = 2 + 0.4 * 0.8 = 2.32
    % y = -0.4 * x + 2.32
    % (0, 2.32) und (3, 1.12)
    \draw[color=cMittelsenkrechten] (0,2.32) -- node {} (3, 1.12);

    % Umkreismittelpunkt
    % x = 1.6
    % y = -0.4*1.6 + 2.32 = 1.68
    \node [circle,inner sep=1pt,fill=cMittelsenkrechten] at (1.6,1.68) {};

    % ((3.2+1.6)/2, (0+4)/2) ) = (2.4, 2) \in Mittelsenkrechte auf a
    % 2 = m * 2.4 + t
    % m_BC = delta y / delta x = (0-4)/(3.2-1.6) = -2.5 = -5/2
    % m = 2/5
    % t = y - m*x = 2 - 2.4 * 2/5  = 1.04
    % => y = 2/5 * x + 1.04
    % (0, 1.04) und (3, 2.24)
    \draw[color=cMittelsenkrechten] (0, 1.04) -- node {} (3, 2.24);

    % Seitenhalbierende
    \draw[color=cSeitenhalbierende,dashed] (A) -- node {} ($(B)!0.5!(C)$);
    \draw[color=cSeitenhalbierende,dashed] (B) -- node {} ($(A)!0.5!(C)$);
    %\draw[color=red, dashed] (C) -- node {} ($(A)!0.5!(B)$);

    % Schwerpunkt
    % x = 1.6
    \node [circle,inner sep=1pt,fill=cSeitenhalbierende] at (1.6,1.34) {};

    % Winkelhalbierende
    % apex angle = 44° => alpha = 68° => 34°
    \draw[color=cWinkelhalbierende, densely dashed] (A) -- node {} (34:4);

    \begin{scope}[shift={($(B)$)}]
        \draw (146:4) node (BEnd) {};
    \end{scope}

    \draw[color=cWinkelhalbierende, densely dashed] (B) -- node {} (BEnd);
    %\draw[color=red, dashed] (C) -- node {} ($(A)!0.5!(B)$);

    % Inkreismittelpunkt
    % x = 1.6
    % tan(68/2) = y/x => y = tan(34)*x = 1.48
    \node [circle,inner sep=1pt,fill=cWinkelhalbierende] at (1.6,1.08) {};


    % Draw the triangle
    \node[isosceles triangle, isosceles triangle apex angle=44,draw,
        inner sep=0pt,anchor=lower side,rotate=90,draw=black,
        line width=1.5pt, minimum height=4cm]
        (triangle) at (1.6,-0.05) {};
  \end{tikzpicture}
\end{preview}
\end{document}
