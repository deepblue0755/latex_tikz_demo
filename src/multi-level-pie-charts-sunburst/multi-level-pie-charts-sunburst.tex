\documentclass[border=5pt]{standalone}
\usepackage{xcolor}
\definecolor{orange1}{HTML}{F1753A}
\definecolor{orange2}{HTML}{F7A13E}
\definecolor{orange3}{HTML}{FC6300}
\definecolor{root}{HTML}{B2B2B2}
\usepackage{tikz}
\usepackage{pgfmath}
\usetikzlibrary{decorations.text, arrows.meta,calc,shadows.blur,shadings}
\renewcommand*\familydefault{\sfdefault} % Set font to serif family

% arctext from Andrew code with modifications:
%Variables: 1: ID, 2:Style 3:box height 4: Radious 5:start-angl 6:end-angl 7:text {format along path} 
\def\arctext[#1][#2][#3](#4)(#5)(#6)#7{

\draw[
    color=white,
    thick,
    line width=1.3pt,
    fill=#2
]
(#5:#4cm+#3) coordinate (above #1) arc (#5:#6:#4cm+#3)
-- (#6:#4) coordinate (right #1) -- (#6:#4cm-#3) coordinate (below right #1) 
arc (#6:#5:#4cm-#3) coordinate (below #1)
-- (#5:#4) coordinate (left #1) -- cycle;
\def\a#1{#4cm+#3}
\def\b#1{#4cm-#3}
\path[
    decoration={
        raise = -0.5ex, % Controls relavite text height position.
        text  along path,
        text = {#7},
        text align = center,        
    },
    decorate
    ]
    (#5:#4) arc (#5:#6:#4);
}

%arcarrow, this is mine, for beerware purpose...
%Function: Draw an arrow from arctex coordinate specific nodes to another 
%Arrow start at the start of arctext box and could be shifted to change the position
%to avoid go over another box.
%Var: 1:Start coordinate 2:End coordinate 3:angle to shift from acrtext box  
\def\arcarrow[#1](#2)(#3)[#4]{
    \draw[thick,-,>=latex,color=#1,line width=1pt,shorten >=-2pt, shorten <=-2pt] 
        let \p1 = (#2), \p2 = (#3), % To access cartesian coordinates x, and y.
            \n1 = {veclen(\x1,\y1)}, % Distance from the origin
            \n2 = {veclen(\x2,\y2)}, % Distance from the origin
            \n3 = {atan2(\y1,\x1)} % Angle where acrtext starts.
        in (\n3-#4: \n1) -- (\n3-#4: \n2); % Draw the arrow.
}

\begin{document}
    \begin{tikzpicture}[
        % Environment Cfg
        font=\sf    \scriptsize,
        % Styles
        myarrow/.style={
            thick,
            -latex,
        },
        Center/.style ={
            circle,
            fill=white,
            text=root,
            align=center,
            font =\footnotesize,
            inner sep=1pt,          
        },
    ]

    % Drawing the center
    \node[Center](ROOT) at (0,0) {Root \\ Node};

    % Drawing the Tex Arcs

    % \Arctext[ID][box-style][box-height](radious)(start-angl)(end-angl){|text-styles| Text}
    % Node 1:   
    \arctext[N1][orange1][15pt](1.5)(200)(90){|\footnotesize\bf\color{white}| Node 1};
        %Sub 1:
        \arctext[N1S1][orange1][13pt](2.5)(120)(90){|\footnotesize\bf\color{white}| Sub1};
            \arctext[N1S1S1][orange1][8pt](3.25)(120)(90){|\footnotesize\bf\color{white}| Sub1-Sub1};
        %Sub 2:
        \arctext[N1S2][orange1][13pt](2.5)(160)(120){|\footnotesize\bf\color{white}| Sub2};
            \arctext[N1S2S1][orange1][8pt](4)(170)(140){|\footnotesize\bf\color{white}| Sub2-Sub1};
            \arctext[N1S2S2][orange1][8pt](4)(140)(110){|\footnotesize\bf\color{white}| Sub2-Sub2};
        %Sub 3:         
        \arctext[N1S3][orange1][13pt](2.5)(200)(160){|\footnotesize\bf\color{white}| Sub3};

    %Node 2:
    \arctext[N2][orange2][15pt](1.5)(-90)(90){|\footnotesize\bf\color{white}| Node 2};
        %Sub 1:
        \arctext[N2S1][orange2][13pt](2.5)(90)(50){|\footnotesize\bf\color{white}| Sub 1};
            \arctext[N2S1S1][orange2][8pt](3.25)(90)(60){|\footnotesize\bf\color{white}| Sub3-Sub1};
            \arctext[N2S1S2][orange2][8pt](3.9)(80)(50){|\footnotesize\bf\color{white}| Sub3-Sub2};
        %Sub 2:
        \arctext[N2S2][orange2][13pt](2.5)(50)(20){|\footnotesize\bf\color{white}| Sub 2};
            \arctext[N2S2S1][orange2][6pt](4.5)(40)(15){|\scriptsize\bf\color{white}| Sub3-Sub1};
            \arctext[N2S2S2][orange2][6pt](4.5)(60)(40){|\scriptsize\bf\color{white}| Sub3-Sub2};
        %Sub 3:
        \arctext[N2S3][orange2][13pt](2.5)(-40)(20){|\footnotesize\bf\color{white}| Sub 3};
            \arctext[N2S3S1][orange2][8pt](3.25)(20)(-10){|\footnotesize\bf\color{white}| Sub3-Sub1};
            \arctext[N2S3S2][orange2][8pt](3.25)(-40)(-10){|\footnotesize\bf\color{white}| Sub3-Sub2};
        %Sub 4: 
        \arctext[N2S4][orange2][13pt](2.5)(-90)(-40){|\footnotesize\bf\color{white}| Sub 4};

    %Node 3:    
    \arctext[N3][orange3][15pt](1.5)(200)(270){|\footnotesize\bf\color{white}| Node 3};
        \arctext[N3S1][orange3][13pt](2.5)(200)(235){|\footnotesize\bf\color{white}| Sub 1};
        \arctext[N3S2][orange3][13pt](2.5)(235)(270){|\footnotesize\bf\color{white}| Sub 2};
            \arctext[N3S2S1][orange3][8pt](4)(250)(270){|\scriptsize\bf\color{white}| Sub2-Sub1};
            \arctext[N3S2S2][orange3][8pt](4)(230)(250){|\scriptsize\bf\color{white}| Sub2-Sub2};

    %Drawing the Arrows
    %\arcarrow(above/below ID)(abobe/below ID)[shift]
    \arcarrow[orange1](below N1S2S2)(above N1S2)[10];
    \arcarrow[orange1](below N1S2S1)(above N1S2)[17];

    \arcarrow[orange2](below N2S2S1)(above N2S2)[10];
    \arcarrow[orange2](below N2S2S2)(above N2S2)[15];
    \arcarrow[orange2](below N2S1S2)(above N2S2)[25];

    \arcarrow[orange3](below N3S2S1)(above N3S2)[-10];
    \arcarrow[orange3](below N3S2S2)(above N3S2)[-10];
    \end{tikzpicture}


\end{document}
