\documentclass[border=5pt]{standalone}

\usepackage{tikz}
\usetikzlibrary{shapes,snakes}
\usepackage{amsmath,amssymb}
\usepackage{verbatim}

\begin{comment}
:Title: Boxes with text and math
:Tags: Text and math

Colorful rounded boxes seems to be an important part of most poster presentations.
Here are a few examples on how to create boxes with text and math using PGF/TikZ.
The easiest solution is to put a minipage environment inside a node.
\end{comment}

\begin{document}



% Define box and box title style
\tikzstyle{mybox} = [draw=red, fill=blue!20, very thick,
    rectangle, rounded corners, inner sep=10pt, inner ysep=20pt]
\tikzstyle{fancytitle} =[fill=red, text=white]

\begin{tikzpicture}
\node [mybox] (box){%
    \begin{minipage}{0.50\textwidth}
        To calculate the horizontal position the kinematic differential
        equations are needed:
        \begin{align}
            \dot{n} &= u\cos\psi -v\sin\psi \\
            \dot{e} &= u\sin\psi + v\cos\psi
        \end{align}
        For small angles the following approximation can be used:
        \begin{align}
            \dot{n} &= u -v\delta_\psi \\
            \dot{e} &= u\delta_\psi + v
        \end{align}
    \end{minipage}
};
\node[fancytitle, right=10pt] at (box.north west) {A fancy title};
\node[fancytitle, rounded corners] at (box.east) {$\clubsuit$};
\end{tikzpicture}%
%
\tikzstyle{mybox} = [draw=blue, fill=green!20, very thick,
    rectangle, rounded corners, inner sep=10pt, inner ysep=20pt]
\tikzstyle{fancytitle} =[fill=blue, text=white, ellipse]
%
\begin{tikzpicture}[transform shape, rotate=10, baseline=-3.5cm]
\node [mybox] (box) {%
    \begin{minipage}[t!]{0.5\textwidth}
        Fermat's Last Theorem states that
        \[
            x^n + y^n = z^n
        \]
        has no non-zero integer solutions for $x$, $y$ and $z$ when $n > 2$.
    \end{minipage}
    };
\node[fancytitle] at (box.north) {Fermat's Last Theorem};
\end{tikzpicture}
%

\end{document}
