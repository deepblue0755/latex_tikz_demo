% works with PGFPlots v1.13 and TikZ v3.0.1a
\documentclass[border=5pt,many]{standalone}
\usepackage{pgfplots}
\usepackage{pgfplotstable}
    \pgfplotsset{compat=1.11}

\usepackage{filecontents}
    \begin{filecontents}{data.txt}
        Time core1 core2 core3 core4 mem
        0   7.847  19.51   18.389  18.943    400.90
        1   6.863  64.706  12.871  30        913.50
        2  10      88       0       0       1215.19
        3  57.576  39       0       0       1691.61
        4   0.99   99       0.99    0       1694.64
        5   0      40.594  60       0       1698.15
        6   0      96.939   3.03    0       1699.55
        7   0      50.495  48.515   0       1700.09
        8   0.99   53      47       0       1703.00
        9   0      28.283  69       3       1696.77
        10 31.313   0       0      67.677   1697.30
        11 15      84       1.01    2.941   2252.78
        12   0     15      14.141  71.717   2249.72
        13  31     27       6.931  37       2249.00
        14  2      13.725  60.606  28       2248.16
        15  9      34.343  41      19       2248.31
        16 32      41.414  25.743   0       2250.18
        17 26      33.663  20.408  21       2249.89
        18 23      13      40      25.253   2249.89
        19 47.525  18.182  22      12.121   2249.60
        20 34.694  25.253  22.772  16.832   2249.32
        21 22       0.99   42.574  37.374   2249.01
        22 12.871  24      12.121  56.436   2251.39
        23 17.172  15.152  49.02   20.202   2252.57
        24 27       5.051  32.653  36       2252.72
    \end{filecontents}

    \begin{filecontents}{data2.txt}
        0
        3
        10
        22
        24
    \end{filecontents}

\begin{document}
    \begin{tikzpicture}

        % define `xmax' value
        % (it has to be a command because it is later needed outside of an
        %  axis environment to filter the `steps' elements, which are greater
        %  than `xmax')
        \def\xmax{24}

        % define color for the vertical lines for the steps
        \colorlet{step color}{black!60}


        % define here what both axis environments have in common
        % so you don't have to repeat this stuff at every axis
        \pgfplotsset{
            every axis/.append style={
                enlargelimits=false,
                width=15cm,
                height=8cm,
                xmin=0,
                xmax=\xmax,
                axis on top,
            },
        }

            %%% collect all time stamps of the steps in `\allX'
            %%% it is later used in the axis environment to draw the lines
            %%% below the axis lines
            % store table for the steps
            \pgfplotstableread[header=true]{data2.txt}{\data}
            % store number of rows
            \pgfplotstablegetrowsof{\data}
                \pgfmathsetmacro{\rows}{\pgfplotsretval-1}
            % store first element to `\allX'
            \pgfplotstablegetelem{0}{[index] 0}\of\data
                \pgfmathsetmacro{\first}{\pgfplotsretval}
            \def\allX{\first}
            % cycle through the rest of the list and append the time to
            % `\allX' if the value is smaller than `\xmax'
            \pgfplotsforeachungrouped \i in {1,...,\rows} {
                \pgfplotstablegetelem{\i}{[index] 0}\of\data
                \pgfmathparse{(\pgfplotsretval<\xmax) ? 1 : 0}
                \ifdim \pgfmathresult pt>0pt
                    \edef\allX{\allX,\pgfplotsretval}
                \fi
            }

        \begin{axis}[
            area style,
            stack plots=y,
            xlabel={Time},
            ylabel={CPU usage},
            ymin=0,
            ymax=150,
            ytick distance=25,  % <-- to match ticks on both axis
        ]

            \foreach \i in {1,...,4}{
                \addplot table [x=Time,y=core\i]{data.txt} \closedcycle;
            }

            % use `ybar interval' plot to fake the vertical fills
            % (for that the `stack plots' has to be turned of to avoid
            %  an error message. Also this has to be plotted _after_ the
            %  `stack plots', because the plot sequence is reversed for
            %  `stack plots' --> last `\addplot' is drawn first)
            \addplot [
                stack plots=false,
                draw=none,
                fill=green!10,
                ybar interval,
            ]
                table [
                    x index=0,
                    % because only every second step should be filled
                    % switch the height of the bar between `ymax' and
                    % `ymin'. (To fill it the other way round, just
                    % replace `ymin' with `ymax' and vice versa)
                    y expr={ifthenelse(mod(\coordindex,2) == 0,
                        \pgfkeysvalueof{/pgfplots/ymax},
                        \pgfkeysvalueof{/pgfplots/ymin})
                    },
                ]
                    {data2.txt};

        \end{axis}

        \begin{axis}[
            no markers,
            %
            %%% draw step labels
            % therefore use the data of the first `\addplot'
            xtick=data,
            % they should be drawn in the middle of two values
            x tick label as interval,
            % define how the label should look like
            xticklabel={
                % because indexing starts at 0 --> add 1
                \pgfmathparse{\ticknum + 1}
                Step \pgfmathprintnumber{\pgfmathresult}
            },
            %
            ymin=0,
            ymax=3000,
            ylabel=memory usage,
            % draw "step" labels on top of graph
            xticklabel pos=upper,
            yticklabel pos=right,% <---
            clip=false,% <---
            % in case steps are larger than `xmax' -->  force it to be `xmax'
            % (here you see how to extract the `xmin' and `xmax' values
            %  when you are _inside_ of an axis environment)
            restrict x to domain*=
                \pgfkeysvalueof{/pgfplots/xmin}:\pgfkeysvalueof{/pgfplots/xmax},
        ]

            % use `ybar interval' plot to fake some vertical lines
            % this also enables the easy printing of the `xticklabels'
            \addplot [
                draw=step color,
                ybar interval,
            ]
                table [
                    x index=0,
                    y expr={ifthenelse(mod(\coordindex,2) == 0,
                        \pgfkeysvalueof{/pgfplots/ymax},
                        \pgfkeysvalueof{/pgfplots/ymin})
                    },
                ]
                    {data2.txt};

            % now draw the other lines regarding to the second y axis
            \addplot [very thick,draw=green] table [x=Time,y=mem] {data.txt};

        \end{axis}
    \end{tikzpicture}
\end{document}
