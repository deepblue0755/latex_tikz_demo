% Table in the shape of an arrow
% Author: Gonzalo Medina
\documentclass{article}
\usepackage{tikz}
%%%<
\usepackage{verbatim}
\usepackage[active,tightpage]{preview}
\PreviewEnvironment{tikzpicture}
\setlength\PreviewBorder{10pt}%
%%%>
\begin{comment}
:Title: Table in the shape of an arrow
:Tags: Matrices;Arrows;Decorations
:Author: Gonzalo Medina
:Slug: arrow-table

This table is drawn using the TikZ matrix library,
in order to get the shape of an arrow.

It was written by Gonzalo Medina on TeX.SE.
Sligh modifications to the original code: sans serif font,
small caps instead of all caps style, indentation and spacing.
\end{comment}
\usetikzlibrary{calc,matrix,decorations.markings,decorations.pathreplacing}

\definecolor{colone}{RGB}{209,220,204}
\definecolor{coltwo}{RGB}{204,222,210}
\definecolor{colthree}{RGB}{207,233,232}
\definecolor{colfour}{RGB}{248,243,214}
\definecolor{colfive}{RGB}{245,238,197}
\definecolor{colsix}{RGB}{243,235,179}
\definecolor{colseven}{RGB}{241,231,163}

\tikzset{ 
  table/.style={
    matrix of nodes,
    row sep=-\pgflinewidth,
    column sep=-\pgflinewidth,
    nodes={rectangle,text width=2cm,align=center},
    text depth=1.25ex,
    text height=2.5ex,
    nodes in empty cells
  }
}

\renewcommand*{\familydefault}{\sfdefault}
\newcommand{\cbox}[1]{\parbox[t]{2cm}{\centering #1}}

\begin{document}

\begin{tikzpicture}
  \matrix (mat) [table] {
    |[fill=colfour]|      & |[fill=colfour]|  & |[fill=colfour]|
      & |[fill=colfour]|  & |[fill=colfour]|  &                   \\
    |[fill=colfive]|      & |[fill=colfive]|  & |[fill=colfive]|
      & |[fill=colfive]|  & |[fill=colfive]|  &                   \\
    |[fill=colsix]|       & |[fill=colsix]|   & |[fill=colsix]|
      & |[fill=colsix]|   & |[fill=colsix]|   & |[fill=colsix]|   \\
    |[fill=colseven]|     & |[fill=colseven]| & |[fill=colseven]|
      & |[fill=colseven]| & |[fill=colseven]| & |[fill=colseven]| \\
    |[fill=colone]|       & |[fill=coltwo]|   & |[fill=colthree]|
      & |[fill=coltwo]|   & |[fill=colone]|   & |[fill=colone]|   \\
    |[fill=colone]|       & |[fill=coltwo]|   & |[fill=colthree]|
      & |[fill=coltwo]|   & |[fill=colone]|   & |[fill=colone]|   \\
    |[fill=colone]|       & |[fill=coltwo]|   & |[fill=colthree]|
      & |[fill=coltwo]|   & |[fill=colone]|   &                   \\
    |[fill=colone]|       & |[fill=coltwo]|   & |[fill=colthree]|
      & |[fill=coltwo]|   & |[fill=colone]|   &                   \\
  };

  % horizontal rules
  \foreach \row in {2,3,4}
    \draw[white] (mat-\row-1.north west) -- (mat-\row-6.north east);
  \draw[white,ultra thick] (mat-1-1.north west) -- (mat-1-6.north east);
  \draw[white,ultra thick] (mat-5-1.north west) -- (mat-5-6.north east);

  % vertical rules
  \foreach \col in {2,3,4,5}
    \draw[white] (mat-5-\col.north west) -- (mat-8-\col.south west);

  % The labels
  \node[fill=colfour] at (mat-1-3) {Firm Infrastructure};
  \node[fill=colfive] at (mat-2-3) {Human Resources Management};
  \node[fill=colsix] at (mat-3-3) {Technology Development};
  \node[fill=colseven] at (mat-4-3) {Procurement};
  \node at ([yshift=-10pt]mat-6-1) {\cbox{Inbound Logistics}};
  \node at ([yshift=-10pt]mat-6-2) {\cbox{Operations \\\mbox{}}};
  \node at ([yshift=-10pt]mat-6-3) {\cbox{Outbound Logistics}};
  \node at ([yshift=-10pt]mat-6-4) {\cbox{Marketing \& Sales}};
  \node at ([yshift=-10pt]mat-6-5) {\cbox{Service \\\mbox{}}};
  \node[rotate = 90] at ([xshift=-52pt]mat-3-1.north)
    {\textsc{Support Activities}};
  \node at ([yshift=-19pt, xshift=-0.5cm]mat-8-3.south)
    {\textsc{Primary Activities}};

  % Erase some visible lines outside the arrow
  \fill[white] (mat-1-5.north east) -- (mat-5-6.north east)
    -- (mat-1-6.north east) -- cycle;
  \fill[white] (mat-8-5.north east) -- (mat-5-6.north east)
    -- (mat-8-6.north east) -- cycle;

  % Draw the arrow tip
  \shade[top color=colfour!70, bottom color=colfour!70,
    middle color=colseven, draw=white, ultra thick] 
    (mat-1-5.north) -- (mat-5-6.north) -- (mat-8-5.south) -- 
    (mat-8-5.south east) -- (mat-5-6.north east) -- (mat-8-5.south east) -- 
    (mat-5-6.north east) -- (mat-1-5.north east) -- cycle;

  % The slanted "Margin" labels
  \begin{scope}[decoration={markings,
    mark=at position .5 with \node[transform shape] {Margin};}]
  \path[postaction={decorate}] 
    ( $ (mat-1-5.north)!0.5!(mat-1-5.north east) $ )
    -- ( $ (mat-5-6.north)!0.5!(mat-5-6.north east) $ );
  \path[postaction={decorate}] 
    ( $ (mat-5-6.north)!0.5!(mat-5-6.north east) $ )
    -- ( $ (mat-8-5.south)!0.5!(mat-8-5.south east) $ );
  \end{scope}

  % The braces
  \draw[decorate, decoration={brace, mirror, raise=6pt}]
    (mat-1-1.north west) -- (mat-5-1.north west);
  \draw[decorate, decoration={brace, mirror, raise=6pt}]
    (mat-8-1.south west) -- (mat-8-5.south);
\end{tikzpicture}
\end{document}
