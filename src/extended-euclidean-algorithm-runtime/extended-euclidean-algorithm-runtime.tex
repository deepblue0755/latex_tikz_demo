\documentclass[varwidth=true, border=5pt]{article}
\usepackage[active,tightpage]{preview}
\usepackage[latin1]{inputenc}
\usepackage{amsmath}
\usepackage{pgfplots}
\pgfplotsset{compat=1.10}
\usepackage{tikz}
\usetikzlibrary{arrows, positioning}
\usepackage{helvet}
\usepackage[eulergreek]{sansmath}

\begin{document}
\begin{preview}
\pgfplotsset{
    colormap={whitered}{
        color(0cm)=(white);
        color(1cm)=(orange!75!red)
    }
}
\begin{tikzpicture}
    \begin{axis}[
        colormap name=whitered,
        clip mode=individual,
        width=10.0cm,
        height=10.0cm,
        % Grid
        grid = major,
        % size
        %xmin= 40,     % start the diagram at this x-coordinate
        %xmax= 90,   % end   the diagram at this x-coordinate
        %ymin= 0,     % start the diagram at this y-coordinate
        %ymax= 60, % end   the diagram at this y-coordinate
        % Legende
        legend style={
            font=\large\sansmath\sffamily,
            at={(0.5,-0.18)},
            anchor=north,
            legend cell align=left,
            legend columns=-1,
            column sep=0.5cm
        },
        % Ticks
        tick align=inside,
        every axis/.append style={font=\large\sansmath\sffamily},
        minor tick style={thick},
        scaled y ticks = false,
        % Axis
        axis line style = {very thick,shorten <=-0.5\pgflinewidth},
        axis lines = middle,
        axis line style = very thick,
        xlabel=$m$,
        x label style={at={(axis description cs:0.5,-0.05)},
                       anchor=north,
                       font=\boldmath\sansmath\sffamily\Large},
        ylabel=$n$,
        y label style={at={(axis description cs:-0.1,0.5)},
                       anchor=south,
                       rotate=90,
                       font=\boldmath\sansmath\sffamily\Large},
        colorbar,
        colorbar style={
            at={(-0.2,0)},
            anchor=south west,
            height=0.25*\pgfkeysvalueof{/pgfplots/parent axis height},
            title={Schritte}  % ADJUST THIS TO YOUR LANGUAGE
        }
        ]
\addplot[
scatter,
only marks,
mark=square*
]
table[col sep=comma,point meta=\thisrow{steps}] {data.csv};
\end{axis}
\end{tikzpicture}
\end{preview}
\end{document}
