\documentclass[varwidth=true, border=2pt]{standalone}

\usepackage{amsmath}
\usepackage{pgfplots}

\def\cdf(#1)(#2)(#3){0.5*(1+(erf((#1-#2)/(#3*sqrt(2)))))}%
% to be used: \cdf(x)(mean)(variance)

\DeclareMathOperator{\CDF}{cdf}

\begin{document}
\begin{tikzpicture}
    \begin{axis}[
        legend pos=north west,
        axis x line=middle,
        axis y line=middle,
        grid = major,
        width=8cm,
        height=6cm,
        grid style={dashed, gray!30},
        xmin=-5.1,     % start the diagram at this x-coordinate
        xmax= 5.1,    % end   the diagram at this x-coordinate
        ymin= 0,     % start the diagram at this y-coordinate
        ymax= 1.1,   % end   the diagram at this y-coordinate
        %axis background/.style={fill=white},
        x label style={at={(axis description cs:0.5,0)},anchor=north},
        y label style={at={(axis description cs:0,.5)},rotate=90,anchor=south},
        xlabel=$x$,
        ylabel=$\Phi_{\mu, \sigma^2}(x)$,
        %xticklabels={-2,-1.6,...,7},
        %yticklabels={-8,-7,...,8},
        tick align=outside,
        minor tick num=-3,
        enlargelimits=false,
        tension=0.08]
        \addplot[domain=-5.2:5.2,smooth,red!70!black,very thick,samples=200,] gnuplot{\cdf(x)(0)(1)};
      \addlegendentry{$\sigma^2 = 1$}
    \end{axis}
\end{tikzpicture}
\end{document}
