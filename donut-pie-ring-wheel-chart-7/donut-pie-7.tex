\documentclass[border=50pt]{standalone}
\usepackage{tikz}
\begin{document}

% Adjusts the size of the wheel:
\def\innerradius{1.8cm}
\def\outerradius{2.2cm}

% The main macro
\newcommand{\wheelchart}[1]{
    % Calculate total
    \pgfmathsetmacro{\totalnum}{0}
    \foreach \value/\colour/\name in {#1} {
        \pgfmathparse{\value+\totalnum}
        \global\let\totalnum=\pgfmathresult
    }

    \begin{tikzpicture}
    % The text in the center of the wheel
      \node[align=center,text width=2*\innerradius]{Ratings given by \pgfmathprintnumber{\totalnum}~participants};

      % Calculate the thickness and the middle line of the wheel
      \pgfmathsetmacro{\wheelwidth}{\outerradius-\innerradius}
      \pgfmathsetmacro{\midradius}{(\outerradius+\innerradius)/2}

      % Rotate so we start from the top
      \begin{scope}[line width=\wheelwidth,rotate=90]

      % Loop through each value set. \cumnum keeps track of where we are in the wheel
      \pgfmathsetmacro{\cumnum}{0}
      \foreach \value/\colour/\name in {#1} {
            \pgfmathsetmacro{\newcumnum}{\cumnum + \value/\totalnum*360}

            % Calculate the percent value
            \pgfmathsetmacro{\percentage}{\value/\totalnum*100}
            % Calculate the mid angle of the colour segments to place the labels
            \pgfmathsetmacro{\midangle}{-(\cumnum+\newcumnum)/2}

            % This is necessary for the labels to align nicely
            \pgfmathparse{
               (-\midangle<5?"south":
                (-\midangle<85?"south west":
                 (-\midangle<105?"west":
                  (-\midangle<175?"north west":
                   (-\midangle<185?"north":
                    (-\midangle<265?"north east":
                     (-\midangle<275?"east":
                      (-\midangle<355?"south east":"south")
                     )
                    )
                   )
                  )
                 )
                )
               )
            } \edef\textanchor{\pgfmathresult}

            % Draw the color segments. Somehow, the \midrow units got lost, so we add 'pt' at the end. Not nice...
            \draw[\colour] (-\cumnum:\midradius pt) arc (-\cumnum:-(\newcumnum):\midradius pt);

            % Draw the data labels
            \node at (\midangle:\outerradius + 1ex) [inner sep=0pt, outer sep=0pt, ,anchor=\textanchor]{\name: \value\ (\pgfmathprintnumber{\percentage}\%)};

            % The 'spokes'
            \foreach \i in {0,...,\value} {
                \draw [gray,thin] (-\cumnum-\i/\totalnum*360:\innerradius) -- (-\cumnum-\i/\totalnum*360:\outerradius);
            }

            % Set the old cumulated angle to the new value
            \global\let\cumnum=\newcumnum
        }

      \end{scope}
      \draw[gray] (0,0) circle (\outerradius) circle (\innerradius);
    \end{tikzpicture}
}

% Usage: \wheelchart{<value1>/<colour1>/<label1>, ...}
\wheelchart{20/green/good,  10/yellow/medium, 9/red/bad, 5/orange/neutral}

\end{document}
