\documentclass{article}

\usepackage[utf8]{inputenc} % this is needed for umlauts
\usepackage[ngerman]{babel} % this is needed for umlauts
\usepackage[T1]{fontenc}    % this is needed for correct output of umlauts in pdf

\usepackage[pdftex,active,tightpage]{preview}
\setlength\PreviewBorder{2mm}
\usepackage{tikz}
\usetikzlibrary{shapes, calc, decorations, automata}
\usepackage{amsmath,amssymb}

\tikzstyle{textnode}=[
    draw=blue!70,   % draw the border with 70% transparent blue
    rectangle,      % the shape of the node is a rectangle
    fill=blue!10,   % fill the box with 20% blue
    text width=4cm,
    inner xsep=3pt,
    text centered]

\tikzstyle{smallnode}=[
    draw=blue!70,   % draw the border with 70% transparent blue
    rectangle,      % the shape of the node is a rectangle
    fill=blue!10,   % fill the box with 20% blue
    text width=1cm,
    text centered]

\begin{document}
\begin{preview}
\begin{tikzpicture}[->,>=stealth,auto, very thick]
        % Draw the vertices.
        \node[textnode] (a) {Pfadüberdeckung};
        \node[textnode, below of=a, node distance=4cm] (b) {Zweigüberdeckung};
        \node[textnode, below of=b] (c) {Anweisungsüberdeckung};

        \node[textnode, right of=a, node distance=5cm] (d) {Mehrfache\\Bedingungsüberdeckung};
        \node[textnode, below of=d, node distance=2cm] (e) {Minimal-mehrfache Bedingungsüberdeckung};
        \node[textnode, below of=e, node distance=2.2cm] (f) {Einfache\\Bedingungsüberdeckung};

        % Legende
        \node[smallnode, below of=c, node distance=1.5cm] (x) {x};
        \node[smallnode, below of=x] (y) {y};
        \draw (x.south) -- (y.north)
            node    [midway,
                    right of=x,
                    node distance=3cm,
                    draw=none,
                    text width=4cm]
                    {Testverfahren x\\subsummiert\\Testverfahren y};

        % Connect vertices with edges and draw weights
        \path (a) edge node {} (b);
        \path (b) edge node {} (c);
        \path (d) edge node {} (e);
        \path (e) edge node {} (f);
        \path (e) edge node {} (b.east);

        \path (x) edge node {} (y);
\end{tikzpicture}
\end{preview}
\end{document}
